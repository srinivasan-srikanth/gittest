\documentclass[11pt,a4paper]{article}

%\usepackage[T1]{fontenc}
%\usepackage[utf8]{inputenc}
%\usepackage{authblk}

\usepackage{amsmath}
\usepackage{amssymb}
\usepackage{amsthm}
\usepackage{mathabx}
\usepackage{fullpage}
\usepackage{mdwlist}
\usepackage{hyperref}
\usepackage{graphicx}

\newtheorem{theorem}{Theorem}
\newtheorem{corollary}[theorem]{Corollary}
\newtheorem{lemma}[theorem]{Lemma}
\newtheorem{observation}[theorem]{Observation}
\newtheorem{proposition}[theorem]{Proposition}
\newtheorem{definition}[theorem]{Definition}
\newtheorem{claim}[theorem]{Claim}
\newtheorem{fact}[theorem]{Fact}
\newtheorem{remark}[theorem]{Remark}

% General Tricks
\newcommand{\prob}[2]{\mathop{\mathrm{Pr}}_{#1}[#2]}
\newcommand{\avg}[2]{\mathop{\textbf{E}}_{#1}[#2]}
\newcommand{\poly}{\mathop{\mathrm{poly}}}
\newcommand{\tm}[1]{\textrm{#1}}
\newcommand{\bb}[1]{\mathbb{#1}}
\newcommand{\bra}[1]{\{#1\}}
\newcommand{\abs}[1]{\left|#1\right|}
\newcommand{\setcond}[2]{\left\{#1\: \middle|\: #2\right\}}
\newcommand{\errnote}[1]{{ \small #1}} 

% Complexity shortcuts
% General Math shortcuts
\newcommand{\F}{\mathbb{F}}
\newcommand{\R}{\mathbb{R}}
\newcommand{\C}{\mathbb{C}}
\newcommand{\naturals}{\mathbb{N}}
\newcommand{\integer}{\mathbb{Z}}
\newcommand{\rational}{\mathbb{Q}}
\newcommand{\ip}[2]{\langle #1, #2 \rangle}

\newcommand{\union}{\cup}
\newcommand{\intersect}{\cap}
\newcommand{\SPS}{\Sigma\Pi\Sigma}
\newcommand{\spsp}[2]{\Sigma\Pi^{#1}\Sigma\Pi^{#2}}
%\newcommand{\sPsP}[2]{\Sigma\Pi^{[#1]}\Sigma\Pi^{\{#2\}}}
\newcommand{\IMM}{\mathrm{IMM}}
\newcommand{\Det}{\mathrm{Det}}
\newcommand{\pder}[2]{\ensuremath{\partial{#1} \over
  \partial{#2}}}
\newcommand{\mc}[1]{\mathcal{#1}}
\newcommand{\Dim}{\mathrm{dim}}
\newcommand{\spd}[3]{\langle \partial_{#2} #1 \rangle_{\leq #3}}
\newcommand{\pd}[2]{\partial_{#2} #1}
\newcommand{\LM}{\mathrm{LM}}
\newcommand{\MSupp}{\mathrm{MSupp}}
\newcommand{\mynorm}[1]{\Vert #1 \Vert}

\newenvironment{myq}{\newline\noindent\begin{minipage}{\textwidth}}{\end{minipage}}

%Cosmetic commands
%\setlength{\parindent}{0pt}
%\setlength{\parskip}{10pt}


%margin comments
\newcommand{\comment}[2]{\marginpar{\tiny{\textbf{#1: }\textit{#2}}}}
\newcommand{\srikanth}[1]{\comment{S}{#1}}
\newcommand{\nutan}[1]{\comment{N}{#1}}
\newcommand{\chandan}[1]{\comment{C}{#1}}

\newcommand{\f}{\mathrm{\Lambda}}
%

\title{Proof of the Hanson-Wright inequality}
\date{}

\begin{document}
\maketitle

\begin{theorem}
\label{thm:HW}
Let $A$ be a symmetric matrix with trace $0$\footnote{I originally wanted to do this proof for the special case when $A$ has zeroes on the diagonal. But that proof breaks down. It turns out that trace $0$ is a nice linear algebraic invariant that is easier to maintain.} (so that the constant term of the polynomial $x^TAx$ is $0$). Then, we have
\[
\prob{x\sim \{-1,1\}^n}{|x^TAx| \geq \lambda} \leq \exp\{-\Omega(\min\{\frac{\lambda^2}{\mynorm{A}_F^2}, \frac{\lambda}{\mynorm{A}_2}\})\}.
\]
\end{theorem}

The proof is based on an upper bound on the $L_2$-norm of the quadratic polynomial $p(x) = x^TAx$. The proof of the above theorem from the lemma below is an easy Markov's inequality (for suitable $k$) and omitted.

\begin{lemma}
\label{lem:Lknorm}
There is some absolute constant $C> 0$ such that for any $k\geq 2$, we have 
\[
\avg{x}{|x^TAx|^k}\leq C^k \sqrt{k}^k \max\{\mynorm{A}_F, \sqrt{k}\mynorm{A}_2\}^k.
\]
\end{lemma}

\begin{proof}
We prove the above lemma for $k= 2^i$. For other $k$, the proof is subsequently easy by using Jensen and using the above for $k' = 2^{\lceil \lg k\rceil}$.

We proceed by induction on $i$. The base case is $i=2$, where we can directly calculate the expectation. We have
\[
\avg{}{(x^TAx)^2} = \sum_{i < j}  (2 A_{i,j})^2 = 4 \sum_{i < j} A_{i,j}^2 = 2 \mynorm{A}_F^2,
\]
which proves the base case as long as $C \geq 1$.

For the induction case, let $k = 2^{i+1}$. We first use a decoupling trick to reduce the problem of bounding the moment of $x^TAx$ to the same problem for $x^TAy$ where $y$ is independent of $x$. 

\begin{claim}[Decoupling]
\label{clm:decouple}
Assume $A$ has trace $0$. Then, $\avg{x}{|x^TAx|^k}\leq 4^k\avg{x,y}{|x^TAy|^k}$.
\end{claim}

Note that the condition that $A$ has trace $0$ is important. If $A$ is the identity matrix, then the above is not true for small $k$ (a factor of $n$ will enter the picture).

\begin{proof}
Note that since $tr(A)= 0$, we have $x^TAx = \sum_{i\neq j}x_ix_j A_{i,j}$.

Imagine that we partition $[n]$ into $S$ and $\overline{S}$ and consider only the expectation of $|x_S^T A_S x_{\overline{S}}|^k$ where $x_S$ is the restriction of $x$ to the co-ordinates in $S$ and $A_S$ is the natural $|S|\times |\overline{S}|$ submatrix of $A$. Note that each term (recall only terms with $i\neq j$ count) in the expansion of $|x^TAx|$ above appears in $|x_S^T A_S x_{\overline{S}}|^k$ with probability at least $1/4$. This indicates that we should be able to bound the expectation of the former with that of the latter. 

We do this as follows. Let $z\in_R \{0,1\}^n$ be the characteristic vector of $S$. For any fixed $x$, we have
\[
\avg{z}{|x_S^T A_S x_{\overline{S}}|^k} = \avg{z}{|\sum_{i, j} x_i x_j A_{i,j}z_i (1-z_j)|^k} \geq |\sum_{i, j} x_i x_j A_{i,j} \avg{z}{z_i (1-z_j)}|^k = \frac{1}{4^k}|x^TAx|^k
\]
Taking expectations w.r.t. $x$ on both sides yields $\avg{x}{|x^TAx|^k} \leq 4^k\avg{x,S}{|x_SA_Sx_{\overline{S}}|^k}$. In particular, we can fix an $S$ so that this inequality holds. 

Now since the random variables in $x_S$ and $x_{\overline{S}}$ are mutually independent, we may replace the latter by $y_{\overline{S}}$ without any problems. Thus, we have
\[
\avg{x}{|x^TAx|^k} \leq 4^k\avg{x_S,y_{\overline{S}}}{|x_SA_Sy_{\overline{S}}|^k} 
\]

We want to compare the expectation on the RHS above to $\avg{x,y}{|x^TAy|^k}$. Note that this expectation also depends on the variables in $x_{\overline{S}}$ and $y_S$ which are independent of the variables in the above expectation. Using Jensen, we get
\[
\avg{x_S,y_{\overline{S}},x_{\overline{S}},y_S}{|x^T A y|^k} \geq \avg{x_S,y_{\overline{S}}}{|\avg{x_{\overline{S}},y_S}{x^T A y}|^k} =  \avg{x_S,y_{\overline{S}}}{|x_S^T A_S y_{\overline{S}}|^k}
\]
which proves the claim.
\end{proof}

So we now bound $\avg{x,y}{|x^TAy|^k}$. To bound this term, we note that for every fixing of $x$, we get a linear function of $y$. Let $x^T A = (a_{x,i})_{i=1}^n$. Then we have
\[
\avg{x,y}{|x^T Ay|^k} = \avg{x,y}{|\sum_{i}a_{x,i}y_i|^k} \leq 2^k \sqrt{k}^k \avg{x}{|\sum_i a_{x,i}^2|^{k/2}} = 2^k \sqrt{k}^k\avg{x}{(x^TA^2 x)^{k/2}}
\]
where the inequality is the Khinchtine-Kahane inequality.

To bound the expectation in the RHS above, we use the Induction hypothesis. We cannot do so directly, since $A^2$ does not have have trace $0$. So we first subtract $(tr(A^2)/n)I$ so that we have trace $0$\footnote{Here, if we were instead shooting for $0$s on the diagonal, we would have to subtract an ugly diagonal matrix. The disadvantage with that is that now it is unclear what happens to $\mynorm{B}_2$.}. We have $x^T A^2 x = x^T B x + tr(A^2) = x^T B x + tr(A^2) = x^T B x + \mynorm{A}_F^2$. Hence, we have 
\begin{align*}
\avg{x}{|x^T A^2 x|^{k/2}} &= \avg{x}{|x^T B x + \mynorm{A}_F^2|^{k/2}}\\
&\leq C^{k/2}2^{k/2} \max\{\avg{x}{|x^TBx|^{k/2}},\mynorm{A}_F^k\}\\
&\leq C^{k/2}2^{k/2} \max\{\sqrt{k/2}\mynorm{B}_F,(k/2) \mynorm{B}_2,\mynorm{A}_F^2\}^{k/2}
\end{align*}
where the first inequality is a consequence of the triangle inequality for $L_k$ and the second inequality is just the induction hypothesis applied to $B$. Now, note that $\lambda_i(B) = \lambda_i(A^2) - \frac{1}{n}\sum_i \lambda_i(A^2)$. In particular, we get $\mynorm{B}_2 = \max_i |\lambda_i(B)|\leq \max_i |\lambda_i(A^2)| = \mynorm{A^2}_2 = \mynorm{A}_2^2$. Similarly, we also have $\mynorm{B}_F^2 = \sum_i \lambda_i(B)^2 = \sum_i (\lambda_i(A^2) - \frac{1}{n}\sum_j \lambda_j(A^2))^2 \leq\footnote{Cauchy Schwarz: Consider the process where we pick $i\in [n]$ at random and let $X = \lambda_i(A^2)$. Then the quantity on the LHS is $n$ times the variance of $X$, which is of course bounded by $n$ times its second moment.} \sum_i \lambda_i(A^2)^2 = \sum_i \lambda_i(A)^4 \leq \mynorm{A}_2^2 \mynorm{A}_F^2$; hence, we have $\mynorm{B}_F \leq \mynorm{A}_F\mynorm{A}_2$.

Putting things together, the RHS in the display above may be bounded by
\begin{align*}
C^{k/2}2^{k/2} \max\{\sqrt{k}\mynorm{A}_F\mynorm{A}_2,k\mynorm{A}_2^2,\mynorm{A}_F^2\}^{k/2} \leq (2C)^{k/2} \max\{\mynorm{A}_F,\sqrt{k}\mynorm{A}_2\}^k
\end{align*}

Putting everything together, we get
\[
\avg{x}{|x^TAx|^k} \leq (2C)^{k/2}8^k \max\{\sqrt{k}\mynorm{A}_F,k\mynorm{A}_2\}^k \leq C^k \max\{\sqrt{k}\mynorm{A}_F,k\mynorm{A}_2\}^k
\]
for $C \geq 128$.

\end{proof}


\end{document}
